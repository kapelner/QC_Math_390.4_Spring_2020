\documentclass[12pt]{article}

\include{preamble}

\newtoggle{professormode}
\toggletrue{professormode} %STUDENTS: DELETE or COMMENT this line



\title{MATH 390.4 / 650.2 Spring 2020 Homework \#4}

\author{Professor Adam Kapelner} %STUDENTS: write your name here

\iftoggle{professormode}{
\date{Due ......., 2020 under the door of KY604\\ \vspace{0.5cm} \small (this document last updated \currenttime~on \today)}
}

\renewcommand{\abstractname}{Instructions and Philosophy}

\begin{document}
\maketitle

\iftoggle{professormode}{
\begin{abstract}
The path to success in this class is to do many problems. Unlike other courses, exclusively doing reading(s) will not help. Coming to lecture is akin to watching workout videos; thinking about and solving problems on your own is the actual ``working out.''  Feel free to \qu{work out} with others; \textbf{I want you to work on this in groups.}

Reading is still \textit{required}. For this homework set, read Chapters 7-9 of Silver's book. You should be googling and reading about all the concepts introduced in class online. This is your responsibility to supplement in-class with \textit{your own} readings.

The problems below are color coded: \ingreen{green} problems are considered \textit{easy} and marked \qu{[easy]}; \inorange{yellow} problems are considered \textit{intermediate} and marked \qu{[harder]}, \inred{red} problems are considered \textit{difficult} and marked \qu{[difficult]} and \inpurple{purple} problems are extra credit. The \textit{easy} problems are intended to be ``giveaways'' if you went to class. Do as much as you can of the others; I expect you to at least attempt the \textit{difficult} problems. 

This homework is worth 100 points but the point distribution will not be determined until after the due date. See syllabus for the policy on late homework.

Up to 7 points are given as a bonus if the homework is typed using \LaTeX. Links to instaling \LaTeX~and program for compiling \LaTeX~is found on the syllabus. You are encouraged to use \url{overleaf.com}. If you are handing in homework this way, read the comments in the code; there are two lines to comment out and you should replace my name with yours and write your section. The easiest way to use overleaf is to copy the raw text from hwxx.tex and preamble.tex into two new overleaf tex files with the same name. If you are asked to make drawings, you can take a picture of your handwritten drawing and insert them as figures or leave space using the \qu{$\backslash$vspace} command and draw them in after printing or attach them stapled.

The document is available with spaces for you to write your answers. If not using \LaTeX, print this document \textit{including this first page} and write in your answers. \inred{I do not accept homeworks which are \textit{not} on this printout.}

\end{abstract}

\thispagestyle{empty}
\vspace{1cm}
NAME: \line(1,0){380}
\clearpage
}



\problem{These are questions about Silver's book, chapters 7-9.  For all parts in this question, answer using notation from class (i.e. $t ,f, g, h^*, \delta, \epsilon, e, t, z_1, \ldots, z_t, \mathbb{D}, \mathcal{H}, \mathcal{A}, \mathcal{X}, \mathcal{Y}, X, y, n, p, x_{\cdot 1}, \ldots, x_{\cdot p}$, $x_{1 \cdot}, \ldots, x_{n \cdot}$, etc.) as well as in-class concepts (e.g. simulation, validation, overfitting, etc.)} % and also we now have $f_{pr}, h^*_{pr}, g_{pr}, p_{th}$, etc from probabilistic classification as well as different types of validation schemes)

\begin{enumerate}

\easysubproblem{Why are flu fatalities hard to predict? Which type of error is most dominant in the models?}\spc{1}

\easysubproblem{In what context does Silver define extrapolation? Give a couple examples of extraordinary prediction failures (by vey famous people who were considered heavy-hitting experts of their time) that were due to reckless extrapolations.}\spc{1}


\easysubproblem{Using the notation from class, define \qu{self-fulfilling prophecy} and \qu{self-canceling prediction}.}\spc{1}


\easysubproblem{Is the SIR model of infectious disease under or overfit? Why?}\spc{1}

\easysubproblem{What did the famous mathematician Norbert Weiner mean by \qu{the best model of a cat is a cat}?}\spc{1}

\easysubproblem{Not in the book but about Norbert Weiner. From Wikipedia: 

\begin{quote}
Norbert Wiener is credited as being one of the first to theorize that all intelligent behavior was the result of feedback mechanisms, that could possibly be simulated by machines and was an important early step towards the development of modern artificial intelligence.
\end{quote}

What do we mean by \qu{feedback mechanisms} in the context of this class?}\spc{2}


\easysubproblem{I'm not going to both asking about the bet that gave Bob Voulgaris his start. But what gives Voulgaris an edge (p239)? Frame it in terms of the concepts in this class.}\spc{1}

Note: I will not ask questions in this assignment about Bayesian calculations and modeling (a large chunk of Chapter 8) as this is the subject of Math 341. It is obviously important in Data Science (that's why Math 341 is a required course in the new major).

\easysubproblem{Why do you think a lot of science is not reproducible?}\spc{1}

\easysubproblem{Why do you think Fisher did not believe that smoking causes lung cancer?}\spc{1}

\easysubproblem{Is the world moving more in the direction of Fisher's Frequentism or Bayesianism?}\spc{1}

\easysubproblem{How did Kasparov defeat Deep Blue? Can you put this into the context of over and underfiting?}\spc{1}

\easysubproblem{Why was Fischer able to make such bold and daring moves?}\spc{1}

\easysubproblem{What metric $y$ is Google predicting when it returns search results to you? Why did they choose this metric?}\spc{1}

\easysubproblem{What do we call Google's \qu{theories} in this class? And what do we call \qu{testing} of those theories?}\spc{1}

\end{enumerate}

\problem{These are questions...}

\begin{enumerate}

\easysubproblem{How did we define \qu{extrapolation} in class?}\spc{1}

\end{enumerate}

\end{document}


\problem{These are questions about multivariate linear model fitting using the least squares algorithm.}

\begin{enumerate}

\hardsubproblem{Derive $\partialop{\c}{\c^\top A \c}$ where $\c \in \reals^n$ and $A \in \reals^{n \times n}$ but \textit{not} symmetric. Get as far as you can.}\spc{8}

\easysubproblem{Given matrix $X \in \reals^{n \times (p+1)}$, full rank and first column consisting of the $\onevec_n$ vector, rederive the least squares solution $\b$ (the vector of coefficients in the linear model shipped in the prediction function $g$). No need to rederive the facts about vector derivatives.}\spc{10}

\intermediatesubproblem{Consider the case where $p = 1$. Show that the solution for $\b$ you just derived is the same solution that we proved for simple regression in Lecture 8. That is, the first element of $\b$ is the same as $b_0 = \ybar - r \frac{s_y}{s_x}\xbar$ and the second element of $\b$ is $b_1 = r \frac{s_y}{s_x}$.} \spc{10}

\easysubproblem{If $X$ is rank deficient, how can you solve for $\b$? Explain in English.} \spc{2}

\hardsubproblem{Prove $\rank{X} =\rank{X^\top X}$.}\spc{6}

\hardsubproblem{Given matrix $X \in \reals^{n \times (p+1)}$, full rank and first column consisting of the $\onevec_n$ vector, now consider cost multiples (\qu{weights}) $c_1, c_2, \ldots, c_n$ for each mistake $e_i$. As an example, previously the mistake for the 17th observation was $e_{17} := y_{17} - \hat{y}_{17}$ but now it would be $e_{17} := c_{17} (y_{17} - \hat{y}_{17})$.  Derive the weighted least squares solution $\b$. No need to rederive the facts about vector derivatives. Hints: (1) show that SSE is a quadratic form with the matrix $C$ in the middle (2) Split this matrix up into two pieces i.e. $C = C^{\half} C^{\half}$, distribute and then foil (3) note that a scalar value equals its own transpose and (4) use the vector derivative formulas.}\spc{10}


\hardsubproblem{If $p=1$, prove $r^2 = R^2$ i.e. the linear correlation is the same as proportion of sample variance explained in a least squares linear model.}\spc{8}

\intermediatesubproblem{Prove that $g(\bracks{1 ~\xbar_1~ \xbar_2~ \ldots~ \xbar_p}) =\bar{y}$ in OLS.}\spc{10}

\intermediatesubproblem{Prove that $\bar{e} = 0$ in OLS.}\spc{10}

\hardsubproblem{If you model $\y$ with one categorical nominal variable that has levels $A, B, C$, prove that the OLS estimates look like $\ybar_A$ if $x = A$, $\ybar_B$ if $x = B$ and $\ybar_C$ if $x = C$. You can choose to use an intercept or not. Likely without is easier.}\spc{10}


\end{enumerate}

\problem{These are questions related to the concept of orthogonal projection, QR decomposition and its relationship with least squares linear modeling.}

\begin{enumerate}

\hardsubproblem{[MA] Prove that if a square matrix is both symmetric and idempotent then it must be an orthogonal projection matrix.}\spc{10}

\easysubproblem{Prove that $I_n$ is an orthogonal projection matrix $\forall n$.}\spc{3}


\easysubproblem{What subspace does $I_n$ project onto?}\spc{3}

\easysubproblem{Consider least squares linear regression using a design matrix $X$ with rank $p + 1$. What are the degrees of freedom in the resulting model? What does this mean?}\spc{6}


\intermediatesubproblem{If you are orthogonally projecting the vector $\y$ onto the column space of $X$ which is of rank $p + 1$, derive the formula for $\proj{\colsp{X}}{\y}$. Is this the same as in OLS?}\spc{6}

\hardsubproblem{We saw that the perceptron is an \textit{iterative algorithm}. This means that it goes through multiple iterations in order to converge to a closer and closer $\w$. Why not do the same with linear least squares regression? Consider the following. Regress $\y$ using $\X$ to get $\yhat$. This generates residuals $\e$ (the leftover piece of $\y$ that wasn't explained by the regression's fit, $\yhat$). Now try again! Regress $\e$ using $\X$ and then get new residuals $\e_{new}$. Would $\e_{new}$ be closer to $\zerovec_n$ than the first $\e$? That is, wouldn't this yield a better model on iteration \#2? Yes/no and explain.}\spc{5}


\intermediatesubproblem{Prove that $\Q^\top = \Q^{-1}$ where $\Q$ is an orthonormal matrix such that $\colsp{\Q} = \colsp{\X}$ and $\Q$ and $\X$ are both matrices $\in \reals^{n \times (p+1)}$. Hint: this is purely a linear algebra exercise.}\spc{6}


\intermediatesubproblem{Prove that the least squares projection $\H = \XXtXinvXt = \Q\Q^\top$.}\spc{10}

\intermediatesubproblem{Prove that an orthogonal projection onto the $\colsp{\Q}$ is the same as the sum of the projections onto each column of $\Q$.}\spc{8}

\easysubproblem{Prove that adding a new column to $\X$ results in SST remaining the same.}\spc{1}





\hardsubproblem{[MA] Prove that $\rank{\H} =\tr{\H}$. Hint: you will need to use facts about eigenvalues and the eigendecomposition of projection matrices that we learned in class.}\spc{10}


\end{enumerate}

\problem{All of these are extra credit. This is for students who want to get a taste of a first year linear model theory class at the graduate level. The prereq to do these problems is Math 368/621. Only attempt these if you have time!

In linear modeling, $\mathcal{H} = \braces{\x \w~:~\w \in \reals^{p+1}}$ where $\x = \bracks{1~x_1~\ldots~x_p}$, a row vector. Thus, there is a best function $h^*(\x) = \x\bbeta$ where $\bbeta = \bracks{\beta_0~\beta_1~\ldots~\beta_p}^\top$, a column vector and $y = h^*(\x) = \x\bbeta + \mathcal{E}$. Imagine that for all $n$ observations in $\mathbb{D}$, the $\Y = X\bbeta + \bv{\mathcal{E}}$ where $\bv{\mathcal{E}} \sim \multnormnot{n}{\zerovec_n}{\sigsq\I_n}$ and $\Y$ is a random vector with dimension $n$ modeling the responses of which $\y$ is a random realization. Assume $\sigsq$ is known. 
}

\begin{enumerate}

\extracreditsubproblem{Show that $\Y\sim \multnormnot{n}{X\bbeta}{\sigsq\I_n}$.}\spc{3}

\extracreditsubproblem{Let $\B = \XtXinv\Xt\Y$, i.e. the r.v. that represents the OLS estimator of which $\b$ is one realization which changes based on the realizations of the error-vector r.v. $\bv{\mathcal{E}}$. Find the distribution of $\B$ and once this is done, its expectation and variance-covariance matrix. Do the entries in $\B$ have dependence?}\spc{3}

\extracreditsubproblem{Find the distribution of $\hat{\Y}$, the vector r.v. of predictions.}\spc{3}

\extracreditsubproblem{Find the distribution of $\bv{E}$, the vector r.v. of residuals.}\spc{3}

\extracreditsubproblem{Find the distribution of $SST$.}\spc{3}

\extracreditsubproblem{Find the distribution of $SSE$.}\spc{3}

\extracreditsubproblem{Find the distribution of $SSR$.}\spc{3}


\extracreditsubproblem{Find the distribution of $R^2$.}\spc{3}

\extracreditsubproblem{Now let $\sigsq$ be unknown. Use the MSE as its estimate. What is the distribution of $\B$ now?}\spc{3}

\extracreditsubproblem{What is the distribution of MSE?}\spc{3}

\extracreditsubproblem{What is the distribution of $R^2$?}\spc{3}

\extracreditsubproblem{Let $\U \sim \multnormnot{n}{\zerovec_n}{\I_n}$ independent of $\V \sim \multnormnot{n}{\zerovec_n}{\I_n}$. Let $\theta$ be the r.v. model of the angle between $\U$ and $\V$. How is $\theta$ distributed?}\spc{3}

\end{enumerate}


\end{document}

%%%%%%%%%%%%%%%%%%FOR HW4


%\hardsubproblem{Trouble in paradise. Prove that the SSE of a multivariate linear least squares model always decreases (equivalently, $R^2$ always increases) upon the addition of a new independent predictor. Keep in mind this holds true even if this new predictor has no information about the true causal inputs to the phenomenon $y$.}\spc{9}

%\intermediatesubproblem{Why is this a bad thing? Explain in English.}\spc{3}

\problem{These are questions related to the concept of validation.}

\begin{enumerate}

\easysubproblem{What is \textit{model validation} and why is it important?}\spc{3}

\easysubproblem{If you are giving a dataset $\mathbb{D}$, what is the problem with truly validating models?}\spc{3}

\easysubproblem{To get around this fundamental problem, we assumed stationarity. Define this term.}\spc{3}

\easysubproblem{Assuming stationarity, how can we do model validation?}\spc{3}

\easysubproblem{What is the cost of this procedure?}\spc{3}

\hardsubproblem{What are some limits of this procedure?}\spc{5}
\end{enumerate}
